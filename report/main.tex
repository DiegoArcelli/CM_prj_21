\documentclass{article}
\usepackage[utf8]{inputenc}
\usepackage{amsmath}
\usepackage{hyperref}
\usepackage{amssymb}

\title{CM project}
\author{Luca Moroni, Diego Arcelli}
\date{December 2021}

\begin{document}

\maketitle

\section{Introduzione}

In questa sezione andremo a definire il problema soggetto del presente elaborato.\\
Partiamo prima però con un richiamo alle notazioni utilizzate e qualche preliminare tecnico.

\subsection{Notazioni}

bla bla bla

\subsection{Il problema}

Il problema da noi selezionato è il numero 32 della sezione "progetti nonML".\\
Il problema consiste nel trovare il minimo di una funzione quadratica, il cui dominio è vincolato, il problema nello specifico rientra nella tipologia "knapsack quadratic non-separable problems".\\
\[\min \{x'Qx + q'x : a'x \geq b, l \leq x \leq u\}\]
Dove $x \in \mathbb{R}^n$ mentre $q, a, l, u$  sono vettori fissati in $\mathbb{R}^n$ con $0 \leq l < u$, e $Q \in \mathbb{R}^{n \times n}$ è una matrice semi-definita positiva.

\section{Algoritmo risolutivo}
Per la risoluzione di tale problema, la traccia ci ha esplicitamente richiesto di utilizzare un metodo iterativo di discesa del gradiente, applicando però la tecnica della gradient projection, in quanto ci troviamo a dover minimizzare una funzione vincolata. Si è deciso di applicare la metodologia di Wolfe, la quale consiste, informalmente, nel calcolare il gradiente in un punto, muoversi nella direzione opposta utilizzando un qualche metodo di scelta del passo, e poi proiettare il punto trovato all'interno della regione ammissibile.\\
Si presentano ora due problematiche, capire come proiettare il punto trovato ad ogni iterazione all'interno della regione ammissibile e capire quando fermarsi, quest'ultima problematica viene risolta definendo come stopping criteria la condizione che la norma della differenza tra la proiezione del punto successivo e il punto attuale sia minore di un certo valore soglia $\epsilon$.\\
Per quanto riguarda invece la tecnica di proiezione la descriveremo nella seguente sottosezione.\\

\subsection{Metodo di proiezione}
Una volta calcolato il punto derivante dal passo di discesa del gradiente si verifica l'appartenenza alla regione ammissibile, nel qual caso la proiezione non avrà alcun effetto, nel caso, invece in cui il punto esca fuori dalla regione ammissibile, definiamo la proiezione come il punto appartenente alla regione ammissibile che minimizza la distanza euclidea dal punto calcolato. Dall'impostazione appena esplicitata ci rendiamo conto che il problema della proiezione è un problema analogo a quello di partenza ma piu semplice, definiamolo formalmente.
\[\min \{\|x - \hat{x} \|_2^2 : a'x \geq b, l \leq x \leq u \}\]
Dove $a, b, l, u$ sono definite come nella formulazione del problema originale, ed $\hat{x}$ è il vettore rappresentante il punto derivante dal passo di discesa del gradiente.\\
Riscriviamo la funzione obiettivo del problema di proiezione.
\[\min \{(x - \hat{x})'I(x - \hat{x}) : a'x \geq b, l \leq x \leq u \}\]
Dove $I$ è la matrice identità. Il problema di Proiezione è definito come Knapsack Separable Quadratic Problem, e per questa specifica tipologia esistono metodi risolutivi che non richiedono la discesa del gradiente. Prima di addentrarci nella descrizione del metodo riformuliamo il problema.\\
Definiamo $\widetilde{x} = x - \hat{x}$ e perciò $x = \widetilde{x} + \hat{x}$ sostituendo abbiamo che, $\widetilde{x} \geq l - \hat{x}$, $\widetilde{x} \leq u - \hat{x}$ definiamo perciò $\widetilde{l} = l - \hat{x}$ e $\widetilde{u} = u - \hat{x}$, inoltre il vincolo $a'x \geq b$ diventa $a'\widetilde{x} \geq b - a'\hat{x}$ volendo inoltre essere consistenti con la formulazione del problema in [2] dobbiamo trasformare tale vincolo nel seguente $-a'\widetilde{x} \leq -(b - a'\hat{x})$ e definiamo $\widetilde{a} = -a$ e $\widetilde{b} = -(b - a'\hat{x})$.
\[\min \{\widetilde{x}'I\widetilde{x} : \widetilde{a}'\widetilde{x} \leq \widetilde{b}, \widetilde{l} \leq \widetilde{x} \leq \widetilde{u} \}\]

Abbiamo ora un problema consistente con la formulazione presente in \cite{Jeong2014IndefiniteKS} e \cite{PATRIKSSON20081}. Facendo sempre riferimento a \cite{Jeong2014IndefiniteKS} e \cite{PATRIKSSON20081}, nei quali articoli sono esplicitate varie metodologie risolutive di un problema del tipo Knapsack Separable Quadratic Problem, abbiamo deciso di utilizzare il Lagrange multiplier search method, chiamato anche break point search method, nello specifico per la ricerca nei break points (che definiremo) applicheremo il metodo tramite sorting, definito anche metodo di ranking.\\
Seguendo \cite{Jeong2014IndefiniteKS} e \cite{PATRIKSSON20081} riformuliamo la derivazione del metodo risolutivo per il nostro problema.\\
Costruiamo la funzione duale lagrangiana (togliamo dalla notazione la tilde per rendere più leggera la notazione)
\[q(\mu) = -b\mu + \sum_j { min_{x_j \in X_j} \{x_j^2 + \mu a_j x_j\}}\]
$x_j \in X_j$ indica che la j-esima componente di $x$ deve essere compresa nel box constrain, stiamo cercando di risolvere il problema applicando la definizione di duale prendendo in considerazione solamente il vincolo di disuguaglianza. Ne segue.
\[q'(\mu) = -b + \sum_j { a_j x_j(\mu)}\]
Dove $x_j(\mu)$ è il minimo (unico) rispetto al box constraint ed è definito come segue.

\begin{equation}
    \begin{cases}
      l_j : \mu \geq -2l_j/a_j\\
      u_j : \mu \leq -2u_j/a_j\\
      x_j : \mu = -2x_j/a_j
    \end{cases}
\end{equation}

Come esplicitato in \cite{Jeong2014IndefiniteKS} possiamo vedere $x_j(\mu) = median \{l_j, u_j, \mu\}$. Definimamo $\mu_j^+ = -2l_j/a_j$ e $\mu_j^- = -2u_j/a_j$ i quali sono punti significativi in quanto, la funzione $q'(\mu)$ è una funzione non cresente lineare a tratti i quali tratti sono delineati dai punti $\mu_j^+$ e $\mu_j^-$. A tal proposito, seguendo sempre \cite{Jeong2014IndefiniteKS} e \cite{PATRIKSSON20081} dobbiamo trovare il valore $\mu^*$ tale per cui $q'(\mu^*) = 0$ il quale ci restituirà il punto ottimo $x(\mu^*)$ dal quale possiamo ricavare il punto proiettato, il quale soddisfa le condizioni KKT necessarie a definirlo un punto di minimo. Per trovare il valore di $\mu^*$, utilizzando il metodo di ranking, ordiniamo dapprima i valori di $\mu_j^+$ e $\mu_j^-$ la quale operazione richiede O(n log n) e tramite una ricerca dicotomica cerchiamo il break point che pone a zero $q'$ oppure troviamo la coppia $\mu_{j^-}$ e $\mu_{j^+}$ tale che $q'(\mu_{j^-}) > 0$ e $q'(\mu_{j^+}) < 0$ i quali ci definiscono un intervallo nel quale la funzione $q'$ è lineare, perciò trovarne lo zero diventa un operazione diretta.\\
Il metodo risolutivo nel suo complesso richiede tempo O(n log n) dove n è il numero di variabili del problema.

\subsection{Metodi di scelta del passo}
Per la selezione del passo dell'algoritmo di discesa di gradiente abbiamo varie scelte, come visto a lezione, nella presente trattazione faremo riferimento ai risultati teorici presenti in \cite{sgd_notes} perciò implementeremo:\\
\begin{itemize}
    \item Constant step size
    \item Diminishing step size
    \item Polyak's step size
    \item Modified Polyak's step size
\end{itemize}

\bibliographystyle{plain}
\bibliography{bibliography.bib}
\end{document}

